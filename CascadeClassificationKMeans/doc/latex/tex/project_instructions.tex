\section{Instrucciones del Proyecto}\label{sec:project-instructions}
El objetivo de este proyecto es el de hace una clasificación en cascada de una imagen a color en formato CIE-Lab, utilizando únicamente los componentes ab de cada pixel.

La clasificación en cascada opera como sigue: en un primer nivel, cada pixel de la imagen se clasificará como perteneciente a dos clases.
Esto se realizará utilizando el algoritmo de k-medias, con un valor de K=2.
A los pilos clasificados en cada una de las dos clases, se clasificarán a su vez en otras dos posibles clases, utilizando también el algoritmo de k-medias.
Se procede de esta manera con cada subclase encontrada hasta que no pueda ser subdividida.
El algoritmo deberá producir un árbol binario en donde en cada iteración, a partir de las hojas del árbol, se pueda generar una imagen que sería una versión simplificada, en cuanto al número de colores, de la imagen original.

El algoritmo de K-medios se debe ejecutar primero utilizando la distancia Euclidiana como métrica, hasta que converja o hasta que el número de píxeles que cambia de clase es lo suficientemente pequeño.
Una vez que ocurra eso, se deberá continuar ejecutando, pero utilizando la distancia de Mahalanobis como métrica.

Se debe entregar el programa funcionando, y un reporte en donde se reporte lo hecho y los resultados obtenidos.
