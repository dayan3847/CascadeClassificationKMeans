\section{Solution}\label{sec:solution}

En esta sección se describe el algoritmo utilizado para resolver el problema planteado.
Al mismo tiempo, se explican los pasos que se deben seguir para poder ejecutar el programa.
Así como un ejemplo de ejecución del programa utilizando una imagen de prueba.

% Preparación del ambiente
\subsection{Preparación del ambiente}\label{subsec:environment-preparation}

Para poder ejecutar el programa, se debe tener instalado el compilador de C++ \texttt{g++} y la herramienta \texttt{make}.
Además, se debe tener instalado el paquete \texttt{libopencv-dev} para poder utilizar la librería \texttt{OpenCV}.

Para instalar \texttt{libopencv-dev} se debe ejecutar el siguiente comando:
\begin{lstlisting}[label={lst:install-opencv}]
sudo apt-get install libopencv-dev
\end{lstlisting}

% Compilación
\subsubsection{Compilación}\label{subsubsec:compilation}

Para compilar el programa, se debe ejecutar el siguiente comando:
\begin{lstlisting}[label={lst:compile}]
make
\end{lstlisting}

% Ejecución
\subsubsection{Ejecución}\label{subsubsec:execution}

El programa tiene una parameter opcional, que es el nombre de la imagen a utilizar.
Si no se especifica el nombre de la imagen, se utilizará la imagen por defecto.
Que es una imagen de prueba de 3x3.

Para ejecutar el programa, se debe ejecutar el siguiente comando:
\begin{lstlisting}[label={lst:run}]
./main
\end{lstlisting}

% Algoritmo
\subsection{Algoritmo}\label{subsec:algorithm}

\begin{enumerate}
    \item Importar imagen a color en formato CIE-Lab.

    Para esto se utiliza la siguiente función:
    \begin{lstlisting}[label={lst:load-image}]
void ImageRepo::byName(
    Mat &image,
    Mat &imageOriginal,
    const string &name
    ) {
    // Read image.
    imageOriginal = imread(name);
    // Convert to float values.
    Mat imageFloat;
    imageOriginal.convertTo(imageFloat, CV_32FC3);
    // Normalize.
    imageFloat /= 255.0;
    // Convert to BGR2Lab.
    cvtColor(imageFloat, image, COLOR_BGR2Lab);
}
    \end{lstlisting}

    \item Se genera una máscara de la misma dimensión que la imagen, que contiene las clases a las que pertenece cada pixel.
    (Inicialmente, todos los píxeles pertenecen a la misma clase \("0"\) ).

    \begin{lstlisting}[label={lst:init-mask}]
// init mask with zeros.
mask = Mat::zeros(image.size(), CV_8UC1);
    \end{lstlisting}

    \item Luego de dividir máscara en dos clases, se procede a dividir cada una de las clases en dos clases más.
\end{enumerate}
