\section{Project instructions}\label{sec:project-instructions}
The goal of this project is to perform a cascaded classification of a color image in CIE-Lab format, using only the ab components of each pixel.
The cascaded classification operates as follows: in the first level, each pixel of the image will be classified as belonging to one of two classes.
This will be done using the k-means algorithm, with a value of K=2.
The pixels classified in each of the two classes will then be classified into two possible classes using the k-means algorithm as well.
This process is repeated with each subclass found until it cannot be subdivided any further.

The algorithm should produce a binary tree where, in each iteration, starting from the leaves of the tree, a simplified version of the original image with fewer colors can be generated.
The k-means algorithm should be executed first using the Euclidean distance as the metric until it converges or until the number of pixels that change class is small enough.
Once this happens, the algorithm should continue running, but using the Mahalanobis distance as the metric.

You should deliver a functioning program and a report detailing what was done and the results obtained.
