\section{Procedure}\label{sec:procedure}
\begin{enumerate}
    \item Importar imagen a color en formato CIE-Lab.

    Para esto se utiliza la siguiente función:
    \begin{lstlisting}
void ImageRepo::byName(Mat &image, Mat &imageOriginal, const string &name) {
    // Read image.
    imageOriginal = imread(name);
    // Convert to float values.
    Mat imageFloat;
    imageOriginal.convertTo(imageFloat, CV_32FC3);
    // Normalize.
    imageFloat /= 255.0;
    // Convert to BGR2Lab.
    cvtColor(imageFloat, image, COLOR_BGR2Lab);
}
    \end{lstlisting}

    \item Se genera una máscara de la misma dimensión que la imagen, que contiene las clases a las que pertenece cada pixel.
    (Inicialmente, todos los píxeles pertenecen a la misma clase \("0"\) ).
    \item Luego de dividir mascarada en dos clases, se procede a dividir cada una de las clases en dos clases más.
\end{enumerate}
